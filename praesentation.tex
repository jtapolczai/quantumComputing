\documentclass{beamer}

\usetheme{Pittsburgh}
\setbeamertemplate{headline}{}

\newcommand{\chapterframe}[1]{\begin{frame}\centering\huge{#1}\end{frame}}
\newcommand{\mt}[1]{\textnormal{#1}}

\title{Quantum Computing}
\subtitle{Introduction \& recent developments}
\author{Stephan Spindler \and Janos Tapolczai \and Dominik Theuerkauf}

\begin{document}

% Titelseite
\begin{frame}
\titlepage
\end{frame}

% Inhalt
% man muss \frame{...} statt \begin{frame}..\end{frame} verwenden, damit
% die TOC erscheint... ist halt so. Magie.
\frame{\frametitle{Contents}\tableofcontents}

% Kapitelframes - nur ein zentriertel Titel
%-----------------------------------------------------------------------------
\section{Mathematics}
\chapterframe{Mathematics}

\begin{frame}{Mathematics}
(hier mathematische einleitung einfuegen)
\end{frame}

% Kapitelframes - nur ein zentriertel Titel
%-----------------------------------------------------------------------------
\section{Algorithms}
\chapterframe{Algorithms}

\begin{frame}{Algorithms}
	\begin{itemize}
		\item Quantum algorithms use a number of techinques, e.g.
			\begin{itemize}
				\item Quantum Fourier Transformation (QFT)
				\item Amplitude Amplification
				\item Quantum Walks
			\end{itemize}
		\item These take $\Omega(2^n)$ time on classical computer,
		\item but often only $O(n^k)$ on quantum computers*.
			\begin{itemize}
				\item * given certain assumptions.
			\end{itemize}
	\end{itemize}
\end{frame}

\begin{frame}{Quantum Fourier Transformation}
	\begin{itemize}
		\item QFT vs classical FT
	\end{itemize}
\end{frame}

\subsection{Shor's algorithm}

\begin{frame}{Shor's Algorithm}{Quantum Fourier Transformation}
	\begin{definition}[Integer factorization]
		$\texttt{factor} : \mathbb{N} \rightarrow \mt{Set}[\mathbb{N}]$\\
		\quad Input: $n \in \mathbb{N}$\\
		\quad Output: $P \subseteq \mathbb{N}$ s.t. $\prod\limits_{p \in P} p = n$.
	\end{definition}

	\begin{itemize}
		\item Best known classical algorithm: generalized prime number sieve (GPNS).
			\begin{itemize}
				\item $O(e^{1.9 \log(n)^{\frac{1}{3}} (\log\log(n))^{\frac{2}{3}}}) = O(e^{f(n)})$ for sub-exponential $f$.
			\end{itemize}
			
		\item Shor's algorithm runs in \textbf{polylogarithmic} time.
			\begin{itemize}
				\item $O(\log(n)^3)$
			\end{itemize}
	\end{itemize}
\end{frame}

\begin{frame}{Amplitude Amplification}
\end{frame}

\subsection{Grover's algorithm}

\begin{frame}{Grover's algorithm}
\end{frame}

\begin{frame}{Quantum Walks}
\end{frame}

\subsection{Element distinctness problem}

\begin{frame}{Element distinctness problem}
\end{frame}



% Kapitelframes - nur ein zentriertel Titel
%-----------------------------------------------------------------------------
\section{Recent developments}
\chapterframe{Recent developments}

\begin{frame}{Recent developments}
(neue papers)
\end{frame}


\end{document}
